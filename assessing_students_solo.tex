%----------------------------------------------------------------------------------------
%	PACKAGES AND OTHER DOCUMENT CONFIGURATIONS
%----------------------------------------------------------------------------------------
% !TEX encoding = UTF-8 Unicode
\documentclass[twoside,twocolumn,a4paper,11pt,english]{article}
\usepackage[utf8]{inputenc}

\usepackage{blindtext} % Package to generate dummy text throughout this template

\usepackage[sc]{mathpazo} % Use the Palatino font
\usepackage[T1]{fontenc} % Use 8-bit encoding that has 256 glyphs
\linespread{1.12} % Line spacing - Palatino needs more space between lines
\usepackage{microtype} % Slightly tweak font spacing for aesthetics

\usepackage[english]{babel} % Language hyphenation and typographical rules

\usepackage[hmarginratio=1:1,left=18mm,right=18mm,top=24mm,columnsep=20pt]{geometry} % Document margins
\usepackage[hang, small,labelfont=bf,up,textfont=it,up]{caption} % Custom captions under/above floats in tables or figures
\usepackage{booktabs} % Horizontal rules in tables

\usepackage{lettrine} % The lettrine is the first enlarged letter at the beginning of the text

\usepackage{enumitem} % Customized lists
\setlist[itemize]{noitemsep} % Make itemize lists more compact

\usepackage{abstract} % Allows abstract customization
\renewcommand{\abstractnamefont}{\normalfont\bfseries} % Set the "Abstract" text to bold
\renewcommand{\abstracttextfont}{\normalfont\small\itshape} % Set the abstract itself to small italic text

\usepackage{titlesec} % Allows customization of titles
\renewcommand\thesection{\Roman{section}} % Roman numerals for the sections
\renewcommand\thesubsection{\roman{subsection}} % roman numerals for subsections
\titleformat{\section}[block]{\large\scshape\centering}{\thesection.}{1em}{} % Change the look of the section titles
\titleformat{\subsection}[block]{\large}{\thesubsection.}{1em}{} % Change the look of the section titles

\usepackage{fancyhdr} % Headers and footers
\pagestyle{fancy} % All pages have headers and footers
\fancyhead{} % Blank out the default header
\fancyfoot{} % Blank out the default footer
\fancyhead[C]{Assessing Knowledge Through Written Reviews $\bullet$ Emil Folino $\bullet$ BTH} % Custom header text
\fancyfoot[RO,LE]{\thepage} % Custom footer text

\usepackage{titling} % Customizing the title section

\usepackage{hyperref} % For hyperlinks in the PDF

%----------------------------------------------------------------------------------------
%	TITLE SECTION
%----------------------------------------------------------------------------------------

\setlength{\droptitle}{-4\baselineskip} % Move the title up

\pretitle{\begin{center}\LARGE\bfseries} % Article title formatting
\posttitle{\end{center}} % Article title closing formatting
\title{Assessing Knowledge Through Written Reviews} % Article title
\author{%
\textsc{Emil Folino} % Your name
\normalsize Blekinge Tekniska Högskola \\ % Your institution
\normalsize \href{mailto:emil.folino@bth.se}{emil.folino@bth.se} % Your email address
}
\date{\today} % Leave empty to omit a date
\renewcommand{\maketitlehookd}{%
\begin{abstract}

\noindent In this paper a method of qualitative assessment of programming student is investigated. The qualitative assessment is done by reading students' review text from individual programming projects and analyzing the content according to the SOLO taxonomy.

\end{abstract}
}

%----------------------------------------------------------------------------------------

\begin{document}

% Print the title
\maketitle

%----------------------------------------------------------------------------------------
%	ARTICLE CONTENTS
%----------------------------------------------------------------------------------------

\section{Introduction}

\lettrine[nindent=0em,lines=3]{Q}uantitatively assessing students in programming and computer science courses is often easy to do. Have the student completed the assigned tasks? Does the application work as it is supposed to do? Does any software tests fail? However assessing the comprehension and understanding of the completed assignments and doing a qualitative evaluation of the student is harder \cite{biggs1982evaluation}. The SOLO Taxonomy was proposed by Biggs and Collis in 1982 and is abbreviated from Structure of the Observed Learning Outcome. The taxonomy is used to qualitatively assess students' work. The SOLO taxonomy consists of five levels of understanding: Prestructural, Unistructural, Multistructural, Relational, and Extended Abstract. In section \ref{sec:examples} the five levels of understanding will be exemplified by the students' review texts.

## More references to quantitative assessments. To show that alot of work has been done there. And not much in qualitative assessment.

There have been limited research in the field of qualitative assessment of programming students. In \cite{mccracken2001multi} McCracken et al. evaluated and first year Computer Science students' programming competency. A framework outlining the expectations of first year Computer Science  students are proposed. However the assessment is only done in a quantitative way and there are no recommendations for any qualitative assessment method.

In \cite{lister2006not} Lister et al. conclude that the experienced programmers answered with SOLO relational responses compared to the novice programmers multi-structural responses. Lister et al. recommend that the students are given written assignments together with programming assignments making it easier to evaluate th understanding and comprehension the students obtains in programming courses.



%------------------------------------------------

\section{Examples of SOLO levels} \label{sec:examples}

In this section examples of how the review texts are mapped to each level of the SOLO taxonomy are shown.

\subsection{Prestructural}

No answer more than repeating the question. The student is failed based on the text. The Prestructural SOLO grade are used for students that have not handed in the projects or have failed to write a review text that explains the problem.

\subsection{Unistructural}

Sökfunktionen finns som egen sida(via navbar) där man kan söka i artiklar och i objektsbeskrivningar med ett ord som utgörs av bokstäver(a-ö) och siffror(0-9). Träffarna presenteras i en lista och första träffen i en artikel/objekt markeras med gult och del av texten.

\subsection{Multistructural}

För varje varv i loopen appendas ett object till salar.json. Så när loopen är färdig så var det bara att lägga till de sista paranteserna och städa upp filen så att det skulle validera som JSON. Jag tyckte detta krav var ganska krångligt och min läsning är absolut inte den snabbaste men den gjorde vad den skulle och det fick vara bra nog.

Jag har försökt att använda mycket inbyggda metoder som '€œmap'€, '€œreduce'€, '€œfilter'€, etc. för att få ut rätt information från de arrays jag använder.

\subsection{Relational}

Valde att inte dela upp min klient som det är gjort i Gomoku. Jag vet att anledningen var för att kunna hålla isär generell och domänspecifik kod, men eftersom jag inte tänkt bygga vidare på den här klienten så lägger jag allt i samma.

När jag bestämde stil för sidan kollade jag runt lite på andra webbplatser som har en koppling till begravningar.

Med tidigare uppgifters klienter som grund gjorde jag en klient som kan testa servern.

\subsection{Extended Abstract}

No examples of Extended Abstract texts were found in the review texts. The students are first-year students and are not asked to come with original material.



%------------------------------------------------

\section{Method}

In our daily teaching at Webbprogramming (dbwebb.se) at BTH the students do programming assignments each week. Together with the exercises they hand in a written review, answering 3-5 questions centered around the topics and assignments of the week. At the end of each study period, a 10 week period, the students hand in an individual project together with an extended review text. The students are graded both with regards to the completed work and the review texts. The following web pages from the program's website explains how the students are graded on their review texts according to the SOLO Taxonomy: \cite{redovisning} and \cite{solo}.


In this report the review texts of three subsequent course projects are analyzed and SOLO graded. The students are given a grade of 1-5 according to the five levels of the SOLO Taxonomy. The SOLO grading will be done manually by reading the review texts and the SOLO grading will be done anonymously, but traceable.

The SOLO grading of the review texts will be done by the author. To ensure an even level of grading the other teachers of the courses are going to do a similar grading and evaluation of a subset of the review texts.

The collection of review texts is done with a web scraper implemented in python. The web scraper fetches the review texts from the students published projects. The review texts are stored in a database together with the website url of the published project and a traceable reference to the student. The review texts are fetched in a manner that removes the names and student acronyms from the review texts to ensure anonymity in most cases.

The analysis of the review texts are done in a web form and the SOLO grade is stored in the same database table as the review texts. The web form removes all styling done by the students and due to the way the collection of review texts are done this further ensures the anonymity of the students.

After the review texts have been analyzed and SOLO graded the SOLO grade will be compared to the final grade of the course. The students' final grade will be fetched and stored in another database table together with the same traceable reference to the students. The final grade for the course and the SOLO grade can now be compared and analyzed to evaluate if there is a correlation between the SOLO grade and the final grade in the course.

As the review texts are taken from three subsequent courses the evolution of the students' understanding of the course material and programming in general can be investigated.

The web scraper and analysis web form can be found at the author's Github page \footnote{https://github.com/emilfolino/pedagogy}.



%------------------------------------------------

\section{Results}

Here results will be shown.



%------------------------------------------------

\section{Discussion}

Here the analysis and discussion of the results will be shown.



%------------------------------------------------

\section{Conclusion}

Here the conclusions are drawn.



%------------------------------------------------

\section{Future Work}


Here I will discuss my plan to do it with NLP and AI.



%------------------------------------------------

\section{Acknowledgements}

Thanks to Mikael Roos, Kenneth Lewenhagen, and Andreas Arnesson of Webbprogrammering at BTH for helping to ensure even SOLO grading by reading and evaluating a subset of the analyzed review texts.



%----------------------------------------------------------------------------------------
%	REFERENCE LIST
%----------------------------------------------------------------------------------------

\bibliographystyle{vancouver}
\bibliography{references}

%----------------------------------------------------------------------------------------

\end{document}
